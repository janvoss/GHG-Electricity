
\documentclass[border=10pt]{standalone}
\usepackage{smartdiagram} %für tikz
\usepackage[english]{babel} %Sprache deutsch (für Silbentrennung)
%\usepackage[utf8]{inputenc} %UTF-Code für Umlaute
%\usepackage[T1]{fontenc} %Trennung von Wörtern mit Umlauten
\usepackage{lmodern} %Schriftart
%\usepackage{libertine}
\renewcommand*\familydefault{\sfdefault} %Serifenlose Schrift als Standard
\usepackage{microtype}
\usesmartdiagramlibrary{additions}
\usetikzlibrary{fit}
\usetikzlibrary{decorations.pathreplacing}

\tikzstyle{container} = [draw, rectangle, semithick, inner sep=0.3cm
]

%Aufzählungsstriche
\AtBeginDocument{
	\def\labelitemi{\normalfont\bfseries{--}}
}


\begin{document}
	
	\begin{tikzpicture}[
		every node/.style = {shape=rectangle, % is not necessary, default node's shape is rectangle
			rounded corners,
			%	draw, semithick,
			text width=3.5cm,
			align=center,
			node distance=0.1cm
		}
		]
		
		%Überschriften
		\node (a)[text depth=.25ex  % Schrift immer auf der gleichen Höhe
		]{\textbf{GHG Target}
		};
		
		
		\node (b)[text depth=.25ex, right = 4 cm of a
		]{\textbf{Prerequisites}
		};
		
		\node (c)[text depth=.25ex, right = 4 cm of b, text width = 6 cm
		]{\textbf{Guiding Principles}
		};
		
		%Spalten
		\node[below= 2.5cm of a, draw, circle %, align=left
		](zero){
			Net zero emissions 
		};
		
		\node[right= 1 cm of zero, %align=left, xshift= -2.5 cm, 
		draw](feasibility){
			Political feasibility 
		};
		
		\node[right= 1cm of feasibility, 
		yshift = -.5 cm,
		%align=left,   xshift= 2.5 cm,
		draw](A){
			Avoid unnecessary cost
		};
		
		\node[above = 1 cm of A, draw, %align=left
		](B){
			Fair burden sharing
		};
		
		\node[above = 1 cm of B, draw, %align=left
		](C){
			Political Commitment 
		};
		
		\node[below = 1 cm of A, draw, %align=left
		](D){
			Competitive Industry 
		};
		
		\node[below= 1 cm of c, align=left, 
		% draw,  
		text width= 6 cm](){
			\begin{itemize}
				\item Technology openness: Enabling all suitable tech paths, including biofuels, e-fuels, H2
				\item 	Cost efficiency
				\item Level playing field
				\item  Resilience
				\item 	Equilibrated market approach: supply and demand side policies
				\item Enabled (instead of planned) transformative processes	
				\item \dots ?
			\end{itemize}
		};
		
		
		%Horizontale Linie unter Überschriften	
		\draw [
		transform canvas={yshift=-0.1cm}
		] (a.south west) -- (c.south east);
		
		%gestrichelte Linie zwischen Bewerten und Verändern
		%\path [draw = none] (b) -- (c) node [midway, yshift = .5cm](Mitte_bc){};
		%\node[below= 5.5cm of Mitte_bc](Mitte_bc2){};
		%\draw [dashed, transform canvas ={xshift=1.5 cm}] (Mitte_bc) -- (Mitte_bc2);
		
		% Zusammenfassung Analyse und Praxis
		
		%	\node[below= 4cm of a, text width=0cm](a-l){};
		%	\node[below= 4cm of b,   text width=0cm](b-r){};
		
		% Pfeile
		
		\draw[->] (zero) -- (feasibility);
		
		\draw[->] (feasibility.east) -- (A.west);
		\draw[->] (feasibility.east) -- (B.west);
		\draw[->] (feasibility.east) -- (C.west);
		\draw[->] (feasibility.east) -- (D.west);
		
		% Geschweifte Klammer
		
		\draw [decorate,decoration={brace,amplitude=10pt, raise=1 cm}] (C.north east) --(D.south east);
		
		
		
		
		%Überschrift
		\path [draw=none] (a.west) -- (c.east) node [midway] (Mittelpunkt) {};	
		\node[above= 1cm of Mittelpunkt, text width=15cm]{\Large Climate Action Target, Prerequisites,  and Guiding Principles};
		
	\end{tikzpicture}
	
\end{document}
